\documentclass[11pt,a4paper]{article}

\usepackage{ngerman,a4wide}
\usepackage[utf8]{inputenc}
\usepackage{comment,xspace}
\usepackage{enumerate,array}
\usepackage{amsthm,amssymb,amsmath}
\usepackage[pdftex]{graphicx}
\usepackage{fancyhdr,lastpage}

\DeclareGraphicsExtensions{.pdf}
\sloppy\parindent=0pt

\newcounter{aufgabe}
\newenvironment{aufgabe}[2][]%
{\stepcounter{aufgabe}\textbf{A\arabic{aufgabe}#1%
\ifx\relax#2\relax. \else~(#2). \fi}}{\par\bigskip\bigskip\goodbreak}

\renewcommand{\labelenumi}{(\alph{enumi})}
\voffset = -2cm
\textheight = 720pt

% ************** Page style *******************
\pagestyle{fancy}
\lfoot{\small\textsf{\datum}}
\cfoot{}
\rfoot{\small\textsf{Seite \thepage/\pageref{LastPage}}}
\renewcommand{\headrulewidth}{0pt}
\renewcommand{\footrulewidth}{0.4pt}

% ************** Commands *********************
\newcommand{\N}{\ensuremath{\mathbb{N}}}
\newcommand{\R}{\ensuremath{\mathbb{R}}}
\newcommand{\Q}{\ensuremath{\mathbb{Q}}}
\newcommand{\C}{\ensuremath{\mathbb{C}}}
\newcommand{\norm}[1]{|\negthinspace|{#1}|\negthinspace|}
\newcommand{\Mat}[1][n\times n]{\ensuremath{\R^{#1}}}
\newcommand{\definedas}{\mathrel{:=}}
\newcommand{\asdefined}{\mathrel{=:}}
\newcommand{\vect}[2]{\begin{pmatrix}	#1\\#2 \end{pmatrix}}
\newcommand{\vectt}[3]{\begin{pmatrix}	#1\\#2\\#3 \end{pmatrix}}
\newcommand{\vecttt}[4]{\begin{pmatrix}	#1\\#2\\#3\\#4 \end{pmatrix}}
\newcommand{\abs}[1]{\lvert #1 \rvert}
\renewcommand{\ker}{\ensuremath{\text{Kern}\,}}
\newcommand{\bild}{\text{Bild}\,}

% ************** Parameter aktuelles Blatt *********
\newcommand{\veranstaltung}{Einführung in künstliche Intelligenz und Machine Learning}
\setcounter{aufgabe}{0}
\newcommand{\blatt}{1}
\newcommand{\datum}{04. Oktober 2019}
\newcommand{\abgabe}{}


\begin{document}\sffamily

\setcounter{page}{1}
%\vspace*{5mm}
\begin{minipage}[l]{10cm}
	{\bfseries\veranstaltung}\\
	Übungsblatt~\blatt\hfill Prof.~Dr.~C.~Möller
	\noindent\rule{\textwidth}{1pt}
\end{minipage}
\hfill
\begin{minipage}[r]{5cm}
\includegraphics[width=5cm]{FK03_CMYK_Linie}
\end{minipage}
\vspace{5mm}


%%%%%%%%%%%%%%%%%%%%%%%%%%%%%%%%%%%%%%%%%%%%%%%%%%%%%%%%%%%%%%%%%%%%%%%%%%%%%
\begin{aufgabe}{Installation der Software}
	Im Rahmen der Lehrveranstaltung werden wir die weit verbreitete, leistungsfähige und einfach handhabbare Python-Bibliothek ``Scikit-Learn'' verwenden. Um die entsprechende Umgebung auf einem Windows-Rechner einzurichten, können Sie folgendermaßen vorgehen:
	\begin{itemize}
		\item Laden Sie Anaconda mit Python 3.7 von www.anaconda.com herunter und installieren Sie es.
		\item Starten Sie den Anaconda Navigator.
	\end{itemize}
\end{aufgabe}

\end{document}
