\documentclass[11pt,a4paper]{article}

\usepackage{ngerman,a4wide}
\usepackage[utf8]{inputenc}
\usepackage{comment,xspace}
\usepackage{enumerate,array}
\usepackage{amsthm,amssymb,amsmath}
\usepackage[pdftex]{graphicx}
\usepackage{fancyhdr,lastpage}

\DeclareGraphicsExtensions{.pdf}
\sloppy\parindent=0pt

\newcounter{aufgabe}
\newenvironment{aufgabe}[2][]%
{\stepcounter{aufgabe}\textbf{A\arabic{aufgabe}#1%
\ifx\relax#2\relax. \else~(#2). \fi}}{\par\bigskip\bigskip\goodbreak}

\renewcommand{\labelenumi}{\Alph{enumi}}
\voffset = -2cm
\textheight = 720pt

% ************** Page style *******************
\pagestyle{fancy}
\lfoot{\small\textsf{\datum}}
\cfoot{}
\rfoot{\small\textsf{Seite \thepage/\pageref{LastPage}}}
\renewcommand{\headrulewidth}{0pt}
\renewcommand{\footrulewidth}{0.4pt}

% ************** Commands *********************
\newcommand{\N}{\ensuremath{\mathbb{N}}}
\newcommand{\R}{\ensuremath{\mathbb{R}}}
\newcommand{\Q}{\ensuremath{\mathbb{Q}}}
\newcommand{\C}{\ensuremath{\mathbb{C}}}
\newcommand{\norm}[1]{|\negthinspace|{#1}|\negthinspace|}
\newcommand{\Mat}[1][n\times n]{\ensuremath{\R^{#1}}}
\newcommand{\definedas}{\mathrel{:=}}
\newcommand{\asdefined}{\mathrel{=:}}
\newcommand{\vect}[2]{\begin{pmatrix}	#1\\#2 \end{pmatrix}}
\newcommand{\vectt}[3]{\begin{pmatrix}	#1\\#2\\#3 \end{pmatrix}}
\newcommand{\vecttt}[4]{\begin{pmatrix}	#1\\#2\\#3\\#4 \end{pmatrix}}
\newcommand{\abs}[1]{\lvert #1 \rvert}
\renewcommand{\ker}{\ensuremath{\text{Kern}\,}}
\newcommand{\bild}{\text{Bild}\,}

% ************** Parameter aktuelles Blatt *********
\newcommand{\veranstaltung}{Einführung in künstliche Intelligenz und Machine Learning}
\setcounter{aufgabe}{0}
\newcommand{\blatt}{1}
\newcommand{\datum}{12. Dezember 2019}
\newcommand{\abgabe}{}


\begin{document}\sffamily

\setcounter{page}{1}
%\vspace*{5mm}
\begin{minipage}[l]{10cm}
	{\bfseries\veranstaltung}\\
	Studienarbeit\hfill Prof.~Dr.~C.~Möller
	\noindent\rule{\textwidth}{1pt}
\end{minipage}
\hfill
\begin{minipage}[r]{5cm}
\includegraphics[width=5cm]{FK03_CMYK_Linie}
\end{minipage}
\vspace{5mm}


%%%%%%%%%%%%%%%%%%%%%%%%%%%%%%%%%%%%%%%%%%%%%%%%%%%%%%%%%%%%%%%%%%%%%%%%%%%%%
\section*{Aufgabenstellung}
Im Rahmen dieser Studienarbeit können Sie sich entweder selbstständig ein Thema wählen oder eines der folgenden bearbeiten. Die entsprechenden Links finden Sie im Moodle-Kurs.
\begin{enumerate}
	\item Bearbeiten Sie die Kaggle Competition "`Reducing Commercial Aviation Fatalities"'.
	\item Bearbeiten Sie die Kaggle Competition "`Bike Sharing Demand"'.
	\item Erstellen Sie ein Modell zur Vorhersage von Gebrauchtwagenpreisen basierend auf dem Datensatz "`Used Cars Dataset"'.
\end{enumerate}
Sollten Sie ein eigenes Thema wählen, so muss dieses mindestens eines der folgenden Modelle beinhalten:
\begin{itemize}
	\item Modell zur Regression
	\item Modell zur Klassifizierung
	\item Modell zur Bildung von Clustern
\end{itemize}

\section*{Anforderungen}
Unabhängig von der gewählten Aufgabe muss die Studienarbeit mindestens folgende Bestandteile enthalten:
\begin{itemize}
	\item Genaue Formulierung der untersuchten Fragestellung inkl. Ziel der Analyse und verwendete Datenquellen
	\item Analyse der Daten inkl. Bewertung der vorliegenden Datenqualität
	\item Erstellung eines Modells zur Bearbeitung der Fragestellung inkl. Begründung der Modellwahl und Beschreibung möglicher Defizite/Verbesserungspotentiale
	\item Diskussion der Resultate in Worten, Zahlen und Grafiken
\end{itemize}

Das Abgabeformat ist ein Jupyter-Notebook, welches in einer Python 3 Umgebung unverändert fehlerfrei laufen muss. D.h. insbesondere das Einlesen der verwendeten Daten muss funktionieren. Dies kann z.B. dadurch sichergestellt werden, dass die Daten durch das Jupyter-Notebook aus einer Online-Quelle geladen werden. Alternativ können die Daten auch aus einem lokalen Verzeichnis eingelesen werden. In diesem Fall ist die exakte Angabe, wo sich welche Quelldatei zu befinden hat  am Anfang des Jupyter-Notebook erforderlich.

\emph{Sollten einzelne (Teile von) Analysen aus externen Quellen verwendet werden, so sind diese Quellen zwingend korrekt zu zitieren.}

\section*{Abgabe}
Die Abgabe des Jupyter-Notebooks und im Falle eines selbst gewählten Themas der Quelldaten (falls nicht per automatischem Download eingebunden) erfolgt per Upload im Moodle-Kurs. Spätester Abgabetermin ist der 23. Januar 2020.

\section*{Bewertung}
Die Bewertung erfolgt anhand der in den folgenden Kategorien erzielten Punkte:
\begin{itemize}
	\item Schwierigkeit der bearbeiteten Fragestellung ("`Überhang"')
	\item Klarheit der Darstellung
	\item Logischer Aufbau
	\item Einsatz von Methoden
	\item Interpretation
\end{itemize}
Dabei wird jede der genannten Kategorien mit 0-5 Punkten bewertet. Die Kategorie "`Einsatz von Methoden"' wird doppelt gewichtet, alle anderen einfach. Insgesamt sind somit maximal 30 Punkte erreichbar, wobei 25 Punkte als 100\% angesehen werden (die Kategorie "`Schwierigkeit der bearbeiteten Fragestellung"' stellt einen Überhang dar).
\vspace{3cm}
\begin{center}
	Viel Freude und Erfolg bei der Erstellung der Studienarbeit!
\end{center}

\end{document}
